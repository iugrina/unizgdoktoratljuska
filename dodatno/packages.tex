%%%%%%%%%%%%%%%%%%%%%%%%%%%%%%%%%%%%%%%%%%%%%%%%%%%%%%%%%%%%%%%%%%%%%%%%%%%%%%%%
% PAKETI
%%%%%%%%%%%%%%%%%%%%%%%%%%%%%%%%%%%%%%%%%%%%%%%%%%%%%%%%%%%%%%%%%%%%%%%%%%%%%%%%

% geometrija stranice
% margine postavljene kao sto i pise u obrascu DR.SC.08
\usepackage[margin=25mm,centering]{geometry} 

% prihvaca tex kod kao UTF-8 omogućavajući
% pisanje znakova s dijakriticima unutar datoteke
% npr. možemo pisati 'č' umjesto č
% PAZITI DA SU DATOTEKE STVARNO SPREMLJENE KAO UTF-8
% nije korisno ako se namjerava koristiti 'latexdiff'
\usepackage[utf8]{inputenc}
% korisnici Windowsa trebaju odkomentirati
% sljedeću liniju te zakomentirati prijašnju
% \usepackage[utf-8]{inputenc}

% omogućiti ispisivanje naših znakova u dokumentu 
\usepackage[T1]{fontenc}

% koristi novu verziju fonta (Latin Modern)
\usepackage{lmodern}

% Palatino font (blizu Arial-a)
%\usepackage[sc]{mathpazo}

% Times New Roman
%\usepackage{mathptmx}
% ili
%\usepackage{txfonts}

% koristan paket kad stvaranja pdf-a
\usepackage[activate={true,nocompatibility},
	final,
	tracking=true,
	kerning=true,
	spacing=true,
	factor=1100,
	stretch=10,
	shrink=10]{microtype}
\microtypecontext{spacing=nonfrench}
%\usepackage{microtype}

% babel !
\usepackage[croatian, english]{babel}
\usepackage{csquotes}

% captioni ce biti centrirani te će im ime
% biti podebljano. Poslije imena tablica (npr. Tablica 1.1:)
% slijedi nova linija. Nakon tablica slijedi mala praznina
\usepackage[center,labelfont=bf]{caption}
	\captionsetup[table]{skip=10pt, labelsep=newline}

% koristan paket za pisanje url-ova u tekstu
\usepackage{url}

% paketi za oblikovanje tablica
\usepackage{booktabs}
\usepackage{multirow}
\usepackage{array}
\newcolumntype{t}{>{\ttfamily}c}

% paketi za oblikovanje lista
\usepackage[ampersand]{easylist}
\usepackage{enumitem}
\usepackage{enumerate}

% boje, bojice, jeee :D
\usepackage{color}
\usepackage{xcolor}
\definecolor{dark-red}{rgb}{0.4,0.15,0.15}
\definecolor{dark-blue}{rgb}{0.15,0.15,0.4}
\definecolor{medium-blue}{rgb}{0,0,0.5}
\definecolor{black}{rgb}{0,0,0}

% paket za stvaranje poveznica u elektronskom
% dokumentu. Koristan kod objavljivanja elektronskog
% oblika dokumenta
\usepackage[final]{hyperref}
%
\PassOptionsToPackage{unicode}{hyperref}
\PassOptionsToPackage{naturalnames}{hyperref}

\hypersetup{
	unicode=true,           % non-Latin characters in Acrobat’s bookmarks
	pdfauthor={Ivo Ugrina},
	pdftitle={Hijerarhijska analiza svojstava nizova znakova metodama 
		znanstvenog računanja i statistike},
	pdfsubject={},
	pdfkeywords={matematika, računarstvo, statistika},
	colorlinks=true,        % false: boxed links; true: colored links
	linkcolor={dark-blue},  % color of internal links (change box color with linkbordercolor)
	citecolor={medium-blue},% color of links to bibliography
	urlcolor={dark-red}     % color of external links
}


% dodatni simboli
\usepackage{pifont}

%
% glossaries
%
\usepackage{supertabular}
\usepackage[xindy,translate=false]{glossaries}
\usepackage{glossary-mcols}
\setglossarystyle{mcolindex}

\loadglsentries{./dodatno/glossary.tex}

\makeglossaries

% prilagođavanje oblika naslova poglavalja (chapter),
% potpoglavlja (section) i odjeljka (subsection)
\usepackage{titlesec}
	\titleformat{\chapter}[display]
	{\normalfont\sffamily\large\centering}
	{\centering\textsc{\chaptertitlename \ \Large \thechapter}}
	{3ex}{\fontfamily{txfonts}\fontsize{30}{20}\selectfont}
	\titleformat{\section}
	{\centering\fontfamily{txfonts}\selectfont}
	{\normalfont\fontsize{26}{16}\selectfont\thesection}{10pt}
	{\fontsize{22}{10}\selectfont}
	\titleformat{\subsection}
	{\centering\fontfamily{txfonts}\selectfont}
	{\normalfont\fontsize{20}{10}\selectfont\thesubsection}{10pt}
	{\fontsize{16}{10}\selectfont}

% crtanje !
\usepackage{tikz}

% verbatim env.
\usepackage{moreverb}

% multi tables/figures
\usepackage{subfig}

\usepackage{float}

% uklucivanje slika
\usepackage{graphicx}
\usepackage{epstopdf}

%
% matematika
%
\usepackage{amssymb ,amsmath, amsthm, amsfonts}
\allowdisplaybreaks[1]
\usepackage{bbm}
% xfrac je koristan paket za umetanje razlomaka unutar teksta
\usepackage{xfrac}

% headeri i footeri
\usepackage{fancyhdr}

% hack za fancyhdr i problem s headerheight-om
% http://nw360.blogspot.com/2006/11/latex-headheight-is-too-small.html
\setlength{\headheight}{15pt}

% paketi za pisanje algoritama
%\usepackage{algorithmic}
\usepackage[croatian,
	linesnumbered,
	onelanguage]{algorithm2e}

% EUROPSKI NACIN RAZDVAJANJA ODLOMAKA
%\setlength{\parskip}{1ex plus 0.5ex minus 0.2ex}
%\setlength{\parindent}{0pt}

% indentacija na prvom paragrafu u poglavlju
\usepackage{indentfirst}

% macroi, naredbe i slicno
\usepackage{xparse}

% paket za stvaranje kazala (indeksa)
\usepackage{makeidx}
\makeindex

% podesi prored na 1.5
\usepackage[onehalfspacing]{setspace}

% dodaje random tekst
\usepackage{lipsum}

% paketi za referenciranje
%\usepackage{varioref}
%\usepackage{cleveref}

%%%%%%%%%%%%%%%%%%%%%%%%%%%%%%%%%%%%%%%%%%%%%%%%%%%%%%%%%%%%%%%%%%%%%%%%%%%%%%%%
% BIBLIOGRAFIJA
%%%%%%%%%%%%%%%%%%%%%%%%%%%%%%%%%%%%%%%%%%%%%%%%%%%%%%%%%%%%%%%%%%%%%%%%%%%%%%%%

% koristiti biber s dodatkom za hrvatski jezik
% kod kompajliranja dokumenta!!!

% paket za upravljanje bibliografijom
%\usepackage[style=authoryear,citestyle=authoryear,doi=false,isbn=false,dashed=false,
%  maxcitenames=2,maxbibnames=100]{biblatex}
\usepackage[doi=true,
	isbn=true,
	url=false,
	maxcitenames=2,
	maxbibnames=100,
	defernumbers=true]{biblatex}
\AtEveryBibitem{\clearfield{note}}

% ovdje dodajemo datoteke koje sadrze bibliografiju
\addbibresource{bibliografija/palindromi.bib}
\addbibresource{bibliografija/palindromi_ostalo.bib}
\addbibresource{bibliografija/ostalo.bib}
\addbibresource{bibliografija/stats.bib}
\addbibresource{bibliografija/stats_ml.bib}
\addbibresource{bibliografija/mojiclanci.bib}

% pogledati 
% https://tex.stackexchange.com/questions/111363/exclude-fullcite-citation-from-bibliography
% za objasnjenje sljedece tri linije
% korisno ako zelimo printati s \fullcite bibligrafiju u zivotopisu
% a taj se clanak ne pojavljuje citiran u ostatku disertacije
\DeclareBibliographyCategory{fullcited}
\newcommand{\mybibexclude}[1]{\addtocategory{fullcited}{#1}}
\mybibexclude{perkovic_multiparameter_2013}


%%%%%%%%%%%%%%%%%%%%%%%%%%%%%%%%%%%%%%%%%%%%%%%%%%%%%%%%%%%%%%%%%%%%%%%%%%%%%%%%
% KORISNE KRATICE (teoremi, definicije, ...)
%%%%%%%%%%%%%%%%%%%%%%%%%%%%%%%%%%%%%%%%%%%%%%%%%%%%%%%%%%%%%%%%%%%%%%%%%%%%%%%%

\newtheoremstyle{stil}{}{}{}{}{\bfseries}{}{ }{}
\theoremstyle{stil}
\newtheorem{tm}{Teorem}[chapter]
\newtheorem{defn}[tm]{Definicija}
\newtheorem{pro}[tm]{Propozicija}
\newtheorem{prop}[tm]{Propozicija}
\newtheorem{lem}[tm]{Lema}
\newtheorem{kor}[tm]{Korolar}

%%%%%%%%%%%%%%%%%%%%%%%%%%%%%%%%%%%%%%%%%%%%%%%%%%%%%%%%%%%%%%%%%%%%%%%%%%%%%%%%
% Izgled stranica (header, footer, ...)

\fancypagestyle{plain}{
	\fancyhf{} 
	\renewcommand{\headrulewidth}{0pt}
	\renewcommand{\footrulewidth}{0pt}
	\fancyfoot[R]{\thepage} 

	% sljedeća linija dodaje u footer informaciju o trenutnoj verziji dokumenta
	% potrebno ju je zakomentirati kod finalnog ispisa
	\fancyfoot[L]{\nouppercase{\footnotesize \today : verzija xxxx}} 
}

\pagestyle{fancy}{
	\fancyhf{} 
	%\renewcommand{\headrulewidth}{0pt}
	%\renewcommand{\footrulewidth}{0pt}
	\rhead{\nouppercase{\footnotesize \leftmark}}
	\fancyfoot[R]{\thepage} 

	% sljedeća linija dodaje u footer informaciju o trenutnoj verziji dokumenta
	% potrebno ju je zakomentirati kod finalnog ispisa
	\fancyfoot[L]{\nouppercase{\footnotesize \today : verzija xxxx}} 
}
\pagestyle{fancy}

% potrebno za dodavanje naslovnice u finalni dokument
\usepackage{pdfpages}

%%%%%%%%%%%%%%%%%%%%%%%%%%%%%%%%%%%%%%%%%%%%%%%%%%%%%%%%%%%%%%%%%%%%%%%%%%%%%%%%
% postavi velicinu razmaka nakon znaka '.' (tocka)
% uvijek na istu vrijednost. Inace ce razmak biti
% veci na kraju recenice nego kod skracenica npr.

%\frenchspacing

%%%%%%%%%%%%%%%%%%%%%%%%%%%%%%%%%%%%%%%%%%%%%%%%%%%%%%%%%%%%%%%%%%%%%%%%%%%%%%%%
% korisno za provjeravanje dokumenta u finalnoj fazi

%\usepackage{refcheck}
%\usepackage{showkeys}
