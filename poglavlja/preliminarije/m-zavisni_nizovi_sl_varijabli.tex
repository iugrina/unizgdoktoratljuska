\section{$m$-zavisni nizovi slučajnih varijabli}

\begin{defn}
	Za niz slučajnih varijabli $(\tsvar{Y}_n)$ kažemo da je 
	\be{m-zavisan}
	\index{$m$-zavisni nizovi slučajnih varijabli}
	ako su za svaki $s \in \mathbb{N}$
	skupovi slučajnih varijabli 
	$\{\tsvar{Y}_1,\cdots,\tsvar{Y}_s\}$
	i $\{\tsvar{Y}_{m+s+1},\tsvar{Y}_{m+s+2},\cdots\}$ nezavisni.
\end{defn}

Primjetimo da je $m$-zavisnost ekvivalentna nezavisnosti
niza slučajnih varijabli za $m=0$.

\begin{defn}
	Za niz slučajnih varijabli $Y_n$ kažemo da je 
	\be{stacionaran}
	\index{stacionaran niz slučajnih varijabli}
	ako za sve $s,t \in \mathbb{N}$
	distribucija slučajnog vektora $(Y_t,\cdots,Y_{t+s})$
	ne ovisi o $t$.
\end{defn}

Drugim riječima, niz je stacionaran ako distribucija
bilo kojih $s$ sljedećih opažanja ne ovisi o vremenu
početka promatranja.

Pretpostavimo da je $(\tsvar{Y}_i)$ stacionaran
niz $m$-zavisnih 
slučajnih varijabli te sa $\mu = \vE{\tsvar{Y}_1}$ označimo očekivanje od $\tsvar{Y}_1$,
sa $\sigma_{00}  = \vVar{\tsvar{Y}_1}$ varijancu od $Y_1$ i
sa $\sigma_{0i}  = \vCov{\tsvar{Y}_t, \tsvar{Y}_{t+i}}$ kovarijancu
između $\tsvar{Y}_t$ i $\tsvar{Y}_{t+i}$. Vrijednosti su dobro
definirane i neovisne o $t$ zbog stacionarnosti niza.
Također, iz $m$-zavisnosti za $i>m$ slijedi $\sigma_{0i}=0$.
Očekivanja i varijance slučajnih varijabli
$\tsvar{S}_n = \sum_{i=1}^{n} \tsvar{Y}_i$ dane su formulama
%
\begin{gather*}
	\vE{\tsvar{S}_n} = n \mu \ ,\\
	\vVar{\tsvar{S}_n} = \sum_{i=1}^{n} \sum_{j=1}^{n} \vCov{\tsvar{Y}_i,\tsvar{Y}_j} =
	n \sigma_{00} + 2(n-1)\sigma_{01} + \cdots
	2(n-m)\sigma_{0m} \ .
\end{gather*}
%
Za $\konv{n}{\infty}$ vrijedi $\konv{\vVar{\tsvar{S}_n}/n}{\sigma^2}$ gdje
je $\sigma^2 = \sigma_{00} + 2\sigma_{01} + \cdots + 2\sigma_{0m}$
pa je po sljedećem teoremu distribucija od $\tsvar{S}_n$ aproksimativno normalna
s očekivanjem $\mu$ i varijancom $\sigma^2$.

\begin{tm}
	\label{tm:mzavisanstacionarnicgt}
	Neka je $\tsvar{Y}_n$ stacionaran $m$-zavisan niz slučajnih varijabli
	s konačnim varijancama te $\tsvar{S}_n = \sum_{i=1}^{n} \tsvar{Y}_i$.
	Tada vrijedi
	\begin{equation}
		\konvpd{
			\frac{\tsvar{S}_n - \vE{\tsvar{S}_n}}{\sqrt{\vVar{\tsvar{S}_n}}}
		}{
			N(0,1),
		}
	\end{equation}
	odnosno
	\begin{equation}
		\konvpd{
			\frac{\tsvar{S}_n - \mu n}{\sqrt{n}}
		}{
			N(0,\sigma^2).
		}
	\end{equation}
\end{tm}
%
U slučaju kada niz nije stacionaran asimptotska distribucija
je također normalna uz određene uvjete. Za iskaz te
tvrdnje potrebna nam je sljedeća definicija.

\begin{defn}
	Za slučajnu varijablu $\tsvar{X}$ definiramo 
	slučajnu varijablu $\tsvar{X}^{\alpha}$ kao
	\[
		\tsvar{X}^{\alpha} =
		\begin{cases}
			X, & \text{za } |X| \leq \alpha \\
			0, & \text{inače}
		\end{cases}
	\]
	Za $\tsvar{X}^{\alpha}$ kažemo da je 
	\be{odrezana}
	\index{odrezana slučajna varijabla}
	 u $\alpha$.
\end{defn}


Sljedeći nam teorem govori o konvergenciji po distribuciji
trokutastog niza $m$-zavisnih slučajnih varijabli. Dokaz se može
naći u \textcite{orey_central_1958}(Teorem 1).

\begin{tm}[Orey] 
\label{Thm_Orey_1958}
\index{Teorem!Orey}
Neka je $(p(n))$ neopadajući niz pozitivnih prirodnih brojeva
koji teži prema beskonačnosti, $m$ pozitivan prirodan broj
te $\tau$ pozitivan realan broj. Ako vrijede sljedeći
uvjeti:
\begin{itemize}
	\item[(1)]{ $\{\tsvar{X}_{nv}\}$,
		$v=1,\ldots,p(n)$, $n=1,\ldots$ je trokutasti
		niz $m$-zavisnih slučajnih varijabli, }
	%
	\item[(2)]{ $\sum_{v=1}^{p(n)} \vE{\tsvar{X}_{nv}^{\tau}} \rightarrow \alpha$ kada 
		$\konv{n}{\infty}$ ($\tsvar{X}_{nv}^{\tau}$ su slučajne varijable
		$\tsvar{X}_{nv}$ odrezane u $\tau$),}
	%
	\item[(3)]{ $\sum_{v=1}^{p(n)} \sum_{w=1}^{p(n)} 
		\vCov{\tsvar{X}_{nv}^{\tau}, \tsvar{X}_{nw}^{\tau}} \rightarrow \sigma^2$ kada
		$\konv{n}{\infty}$ ,}
	%
	\item[(4)]{ $\sum_{v=1}^{p(n)} 
		\vP(|\tsvar{X}_{nv}| > \epsilon ) \rightarrow 0$  za svaki $\epsilon > 0$ ,}
	%
	\item[(5)]{ $\sum_{v=1}^{p(n)} \vVar{\tsvar{X}_{nv}^{\tau}} = O(1)$ kada
		$\konv{n}{\infty}$ ,}
\end{itemize}
tada za $n\to\infty$ vrijedi
\[ 
	\konvpd{ \sum_{v=1}^{p(n)} \tsvar{X}_{nv} }{  N(\alpha,\sigma^2) } \ .
\]
\end{tm}

Opširniji pregled poznatih rezultata,
kao i dokazi odgovarajućih tvrdnji, mogu
se naći u ~\textcite{ferguson_course_1996}.
