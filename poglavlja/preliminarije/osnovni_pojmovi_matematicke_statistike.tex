\section{Osnovni pojmovi matematičke statistike}

Neka je $\Omega$ neprazan skup, $\mathcal{F}$ $\sigma$-algebra
nad skupom $\Omega$ te $\mathcal{P}$ familija dopuštenih
vjerojatnosnih razdioba nad $(\Omega, \mathcal{F})$
indeksirana nekim parametrom $\theta$:
\[
	\mathcal{P} = \left\{ P_{\tv{\theta}} : \tv{\theta} \in \Theta \right\}
\]
gdje je $\Theta \subset \mathbb{R}^k,$ $\Theta \neq \emptyset$
skup svih dopuštenih vrijednosti
parametra $\theta$ neke vjerojatnosne razdiobe.
Uređenu trojku
$(\Omega, \mathcal{F}, \mathcal{P})$
zovemo 
\be{statistički eksperiment}
\index{statistički eksperiment}
ili
\be{statistički model}.
\index{statistički model}

\begin{defn}
	Slučajni vektor $\tsvek{X}=(\tsvar{X}_1,\ldots,\tsvar{X}_n)$
	čije su komponente nezavisne
	slučajne varijable sa zajedničkom vjerojatnosnom razdiobom,
	tj. za čiju funkciju distribucije vrijedi
	\[
		\vP_\theta (\tsvar{X}_1 \leq x_1, \ldots, \tsvar{X}_n \leq x_n) =
		\vP_\theta (\tsvar{X}_1 \leq x_1) \cdots \vP_\theta(\tsvar{X}_n \leq x_n) \ ,
		\ \forall \theta \in \Theta , \ \forall (x_1,\ldots,x_n) \in \setR^n
	\]	
	nazivamo \be{slučajnim uzorkom}
	\index{slučajni uzorak}
	veličine $n$.
\end{defn}


\begin{defn}
	\be{Procjenitelj}
	\index{procjenitelj}
	nepoznatog parametra $\theta$ je 
	slučajna varijabla $\hat{\tsvar{T}}$, definirana kao 
	funkcija slučajnog uzorka $\tsvek{X}$. Piše se
	\[
		\hat{\tsvar{T}} = h(\tsvek{X}) = h(\tsvar{X}_1,\ldots,\tsvar{X}_n) \ ,
	\]
	gdje je $h$ realna funkcija $n$ realnih varijabli.
\end{defn}

Slučajnu varijablu kao funkciju slučajnog uzorka uobičajeno je
u matematičkoj statistici zvati 
\be{statistikom}
\index{statistika}
pa se može reći da je procjenitelj statistika.

\begin{defn}
	Procjenitelj koji zadovoljava $\mathrm{E}_{\theta} \left(\hat{\tsvar{T}}\right) = \theta$
	zovemo \be{nepristrani procjenitelj}.
	\index{procjenitelj!nepristran}
\end{defn}

Budući da veličina uzorka u primjenama teorije statističkog
zaključivanja  može varirati korisno je
za procjenitelj uvesti i oznaku
\[
	\hat{\tsvar{T}}_n = h(\tsvar{X}_{1},\ldots,\tsvar{X}_{n}),\ n \in \mathbb{N} \ ,
\]
kako bi se istakla i ovisnost o veličini $n$.  Prijašnjom je relacijom
ustvari definiran beskonačan niz slučajnih varijabli
$(\hat{\tsvar{T}}_{n})_{n \in \mathbb{N}}$.

\begin{defn}
	Neka je $\left\{ P_\theta: \theta \in \Theta \right\}$
	familija distribucija te $\tsvek{X} = \left\{ \tsvar{X}_1,\tsvar{X}_2,\ldots \right\}$
	beskonačan slučajan uzorak.
	Za niz procjenitelja $(\tsvar{T}_{n})$ parametra $g(\theta)$ kažemo da je 
	\be{konzistentan}
	\index{procjenitelj!konzistentan}
	za parametar $g(\theta)$ ako vrijedi
	\[
		\hat{T}_{n} \overset{P_\theta}{\longrightarrow} g(\theta), \ 
		\forall \theta \in \Theta \ .
	\]
\end{defn}

%\textcolor{red}{Treba li objašnjavati:
%Kolmogorov-Smirnov test, Chi-kvadrat test, p-vrijednost ???}

\noindent
Opširniji uvod u matematičku statistiku može se naći u
\textcite{pausestatistika} ili \textcite{schervish_theory_1995}.
