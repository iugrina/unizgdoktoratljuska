\section{Konvergencija slučajnih vektora i varijabli}

Za $d$-dimenzionalan slučajan vektor 
$\tsvek{X} = (\tsvar{X_1},\cdots,\tsvar{X_d})$
\be{funkciju distribucije od $\tsvek{X}$}
\index{funkcija distribucije slučajnog vektora}
definiramo sa
$F_{\tsvek{X}}(x) = 
\vP( \tsvek{X} \leq \tv{x} ) =
\vP( \tsvar{X_1} \leq x_1, \cdots, \tsvar{X_d} \leq x_d)$
za sve $\tv{x}=(x_1,\cdots,x_d) \in \mathbb{R}^d$. 
\be{Euklidsku normu}
\index{euklidska norma}
od $\tv{x}=(x_1,\cdots,x_d) \in \mathbb{R}^d$
označavamo s $|\tv{x}| = (x_{1}^{2}+\cdots+x_{d}^{2})^{\sfrac{1}{2}}$.

\begin{defn}
	Za niz slučajnih vektora $(\tsvek{X}_n)$ kažemo da 
	konvergira slučajnom vektoru $\tsvek{X}$
%
\begin{itemize}
	\item[(i)]{\be{gotovo sigurno (g.s.)}
		\index{konvergencija!gotovo sigurno}
		ako vrijedi
		\begin{equation} \label{equ:konvgs}
			\vP( \lim_{n \rightarrow \infty} \tsvek{X}_n = \tsvek{X}) = 1\ .
		\end{equation}
		Tada pišemo $\konvgs{\tsvek{X}_n}{\tsvek{X}}$.
	}
	\item[(ii)]{\be{u srednjem reda r}
		\index{konvergencija!u srednjem reda r}
		ako za neki realan broj $r>0$ te za $\konv{n}{\infty}$ vrijedi
		\begin{equation}\label{equ:konvusr}
			\vE{| \tsvek{X}_n - \tsvek{X} |^r}  \rightarrow 0\ .
		\end{equation}
		Tada pišemo $\konvusr{\tsvek{X}_n}{\tsvek{X}}{r}$.
	}
	\item[(iii)]{\be{po vjerojatnosti}
		\index{konvergencija!po vjerojatnosti}
		ako za svaki $\varepsilon >0$ te za $\konv{n}{\infty}$ vrijedi
		\begin{equation}\label{equ:konvpv}
			\vP \left( | \tsvek{X}_n - \tsvek{X} | > 
			\varepsilon \right) \rightarrow 0 \ .
		\end{equation}
		Tada pišemo $\konvpv{\tsvek{X}_n}{\tsvek{X}}$.
	}
	\item[(iv)]{\be{po distribuciji}
		\index{konvergencija!po distribuciji}
		ako za sve točke $x$ u kojima je funkcija distrubucije
		$F_{\tsvek{X}}$ slučajnog vektora $\tsvek{X}$ 
		neprekidna (u oznaci $\tv{x} \in C(F_{\tsvek{X}})$) vrijedi
		\begin{equation} \label{equ:konvpd}
			F_{\tsvek{X}_n}(\tv{x}) \rightarrow F_{\tsvek{X}}(\tv{x})
		\end{equation}
		kada $n \rightarrow \infty$. Tada pišemo $\konvpd{\tsvek{X}_n}{\tsvek{X}}$.
	}
\end{itemize}
\end{defn}

Razlog isključivanja svih točaka u kojima $F_{\tsvek{X}}$ nije neprekidna, kod
konvergencije po distribuciji, možda se čini viškom. Međutim,
potreban (koristan) je što prikazuje sljedeći primjer.

\begin{primjer_}
	Kažemo da je slučajan vektor $\tsvek{X} \in \mathbb{R}^d$
	\be{koncentriran}
	\index{koncentriran slučajan vektor}
	u točki $\tv{c} \in \mathbb{R}^d$ ako je $\vP( \tsvek{X} = \tv{c} )=1$.
	Neka je $(\tsvar{X}_n) \in \mathbb{R}$  niz slučajnih varijabli
	koncentriranih u točkama $\sfrac{1}{n}$ za $n=1,2,\ldots$
	te neka je $\tsvar{X} \in \mathbb{R}$ koncentrirana u 0. Budući da
	$\sfrac{1}{n}$ konvergira prema 0 kada $\konv{n}{\infty}$, moglo bi se
	očekivati da će vrijediti $\konvpd{\tsvar{X}_n}{\tsvar{X}}$.

	Funkcija distribucije od $\tsvar{X}_n$ dana je sa 
	$F_{\tsvar{X}_n} = \mathbbm{1}_{[\sfrac{1}{n},\infty]}(x)$,
	a od $\tsvar{X}$ sa $F_{\tsvar{X}} (x) = \mathbbm{1}_{[0,\infty]}(x)$.
	Očito vrijedi
	$\konv{F_{\tsvar{X}_n}(x)}{F_{\tsvar{X}}(x)}$ za $x \neq 0$, dok za $x=0$ vrijedi
	$F_{\tsvar{X}_n}(0) = 0$ i $F_{\tsvar{X}} (x) = 1$. Budući da $x \notin C(F_X)$
	uvjeti konvergencije po distribuciji su zadovoljeni.
\end{primjer_}

Sljedeća lema daje karakterizaciju konvergencije gotovo sigurno
iz koje se lakše vidi razlika između konvergencije
gotovo sigurno i konvergencije po vjerojatnosti.

\begin{lem}
	$\konvgs{\tsvek{X}_n}{\tsvek{X}}$ ako i samo ako za svaki
	$\varepsilon >0$ vrijedi
	\begin{equation} \label{equ:karakterizacijakonvgs}
		\vP( |\tsvek{X}_k - \tsvek{X}| < \varepsilon,\ \forall k \geq n) \rightarrow 1
	\end{equation}
	kada $\konv{n}{\infty}$.
\end{lem}

Riječima razliku možemo opisati na sljedeći način:
za konvergenciju po vjerojatnosti potrebno je da za svaki $\varepsilon >0$
vjerojatnost da će se $\tsvek{X}_n$ nalaziti unutar $\varepsilon$ odmaka od 
$\tsvek{X}$ teži
prema 1 dok je kod konvergencije gotovo sigurno potrebno da za svaki
$\varepsilon >0$ vjerojatnost da $\tsvek{X}_k$ ostane unutar $\varepsilon$ odmaka
od $\tsvek{X}$ za svaki $k \geq n$ teži prema 1 kada $n$ teži prema beskonačnosti.

Odnose među tipovima konvergencija slučajnih varijabli daje sljedeći teorem.

\begin{tm} \label{tm:vezemedjukonvergencijama}
	Za niz slučajnih vektora $(\tsvek{X}_n)$ i slučajan vektor $\tsvek{X}$ vrijedi
	\begin{itemize}
		\item[(i)]{ $\konvgs{\tsvek{X}_n}{\tsvek{X}} 
			\implies \konvpv{\tsvek{X}_n}{\tsvek{X}}$ }
		\item[(ii)]{ $\konvusr{\tsvek{X}_n}{\tsvek{X}}{r}
			\implies \konvpv{\tsvek{X}_n}{\tsvek{X}}$ }
		\item[(iii)]{ $\konvpv{\tsvek{X}_n}{\tsvek{X}} 
			\implies \konvpd{\tsvek{X}_n}{\tsvek{X}}$ }
	\end{itemize}
\end{tm}

Obrati ne vrijede nužno kao što pokazuju sljedeći primjeri.

\begin{primjer_}
	Po definiciji konvergencije po distribuciji slučajne varijable $\tsvar{X}_n$
	ne moraju biti definirane na istom vjerojatnom prostoru
	kao i njihov limes $\tsvar{X}$ pa time ne možemo ni računati
	uvjet konvergencije po vjerojatnosti (\ref{equ:konvpv}).

	Čak i ako su $\tsvar{X}_n$ i $\tsvar{X}$ definirane na istom vjerojatnosnom prostoru
	konvergencija po distribuciji ne mora povlačiti konvergenciju po
	vjerojatnosti. Neka su $\tsvar{X}_n \sim N(0,1)$  nezavisne i jednako distribuirane
	($\tsvar{X}_n$ su normalne slučajne varijable s očekivanjem 0 i varijancom jednakom 1).
	Tada vrijedi $\konvpd{\tsvar{X}_n}{\tsvar{X}_1}$ dok
	$\nkonvpv{\tsvar{X}_n}{\tsvar{X}_1}$.
\end{primjer_}

\begin{primjer_}
	Neka je $\tsvar{Z}$ slučajna varijabla uniformno distribuirana na intervalu $(0,1)$,
	odnosno $\tsvar{Z} \sim U([0,1])$ i neka su 
	$\tsvar{X}_1=1$,
	$\tsvar{X}_2 = \mathbbm{1}_{[0,\sfrac{1}{2}]}(\tsvar{Z})$,
	$\tsvar{X}_3 = \mathbbm{1}_{[\sfrac{1}{2},1]}(\tsvar{Z})$,
	$\tsvar{X}_4 = \mathbbm{1}_{[0,\sfrac{1}{4}]}(\tsvar{Z})$,\ldots.
	Odnosno, ako je $n=2^k + m$, gdje je $0 \leq m \leq 2^k$ i $k \geq 0$, tada
	je $\tsvar{X}_n = \mathbbm{1}_{[m 2^{-k}, (m+1) 2^{-k})}(Z)$.
	Očito $(\tsvar{X}_n)$ ne
	konvergira za bilo koji $Z \in [0,1)$ te time 
	$\nkonvgs{\tsvar{X}_n}{\tsvar{X}}$.
	Međutim, $\konvusr{\tsvar{X}_n}{0}{r}$ za svaki $r>0$ i $\konvpv{\tsvar{X}_n}{0}$.
\end{primjer_}

\begin{primjer_}
	Neka su $\tsvar{Z} \sim U([0,1])$ i $\tsvar{X}_n = 2^n \mathbbm{1}_{[0,\sfrac{1}{n}]}(\tsvar{Z})$.
	Tada $\vE{ |\tsvar{X}_n|^r } = \frac{2^{nr}}{n} \rightarrow \infty$ pa vrijedi
	$\nkonvusr{\tsvar{X}_n}{0}{r}$ za svaki $r>0$. Međutim, $\konvgs{\tsvar{X}_n}{0}$
	($\{ \lim_{n\rightarrow \infty} \tsvar{X}_n = 0\} = \{\tsvar{Z} >0 \}$ i
	$\vP(\tsvar{Z}>0) = 1$) i $\konvpv{\tsvar{X}_n}{0}$
	(ako je $0 < \varepsilon < 1$, $\vP(|\tsvar{X}_n| > \varepsilon) =
	P(\tsvar{X}_n = 2^n) = \sfrac{1}{n} \rightarrow 0)$.
\end{primjer_}

Iako obrati u Teoremu \ref{tm:vezemedjukonvergencijama}
ne vrijede nužno, postoje dovoljni uvjeti kada obrati vrijede.
Isti su dani sljedećim teoremom. 

\begin{tm} Neka je $\tsvek{c}$ koncentriran slučajni vektor u $\mathbb{R}^d$.
	Tada vrijedi
	\begin{itemize}
		\item[(i)]{ $\konvpd{\tsvek{X}_n}{\tsvek{c}} \implies
			\konvpv{\tsvek{X}_n}{\tsvek{c}}$. }
		\item[(ii)]{Ako $\konvgs{\tsvek{X}_n}{\tsvek{X}}$ i 
			$|\tsvek{X}_n|^r \leq \tsvar{Z}$
			za neki $r>0$ i slučajnu varijablu $\tsvar{Z}$ sa
			$\vE{\tsvar{Z}} < \infty$
			tada vrijedi $\konvusr{\tsvek{X}_n}{\tsvek{X}}{r}$.}
		\item[(iii)]{[Scheff\'{e} (1947.)] Ako $\konvgs{\tsvek{X}_n}{\tsvek{X}}$,
			$\tsvek{X}_n \geq 0$ i 
			$\konv{\vE{\tsvek{X}_n}}{\vE{\tsvek{X}}} < \infty$ tada
			$\konvusr{\tsvek{X}_n}{\tsvek{X}}{1}$. }
		\item[(iv)]{$\konvpv{\tsvek{X}_n}{\tsvek{X}}$ 
			ako i samo ako svaki niz prirodnih
			brojeva $(a_k)$ ima podniz $(a_{p(k)})$
			takav da $\konvgs{\tsvek{X}_{a_{p(k)}}}{\tsvek{X}}$
			kada $\konv{k}{\infty}$. }
	\end{itemize}

\end{tm}

Za funkciju $g:\setR^d \rightarrow \mathbb{R}$ kažemo da
\be{iščezava}
\index{iščezavajuća funkcija na kompaktu}
izvan kompakta ako postoji kompaktan skup $K \subset \mathbb{R}^d$
takav da vrijedi $g(\tv{x})=0$ za svaki $\tv{x} \notin K$.

Veza između konvergencije po distribuciji niza slučajnih vektora
i konvergencije po očekivanju funkcija tih vektora dana je
sljedećim teoremom.

\begin{tm}
	Sljedeći su uvjeti ekvivalentni
	\begin{itemize}
		\item[(i)]{$\konvpd{\tsvek{X}_n}{\tsvek{X}}$}
		\item[(ii)]{$\konv{\vE{g(\tsvek{X}_n)}}{\vE{g(\tsvek{X})}}$ za svaku neprekidnu
		funkciju $g$ koja iščezava izvan kompakta.}
		\item[(iii)]{$\konv{\vE{g(\tsvek{X}_n)}}{\vE{g(\tsvek{X})}}$ za svaku
		ograničenu neprekidnu funkciju $g$.}
		\item[(iv)]{$\konv{\vE{g(\tsvek{X}_n)}}{\vE{g(\tsvek{X})}}$ za svaku ograničenu izmjerivu funkciju
			$g$ takvu da vrijedi $\vP(\tsvek{X} \in C(g))=1$.}
	\end{itemize}
\end{tm}

Nekoliko korisnih pravila za rad s konvergencijom po vjerojatnosti
dano je sljedećom propozicijom.

\begin{pro}
	\label{pro:pravilazakonvpv}
	Pretpostavimo da za nizove slučajnih varijabli
	$(\tsvar{X}_n)$ i $(\tsvar{Y}_n)$
	i neke
	konstante $a,b \in \mathbb{R}$ vrijedi
	$\konvpv{\tsvar{X}_n}{a}$, $\konvpv{\tsvar{Y}_n}{b}$ 
	te neka je $c \in \mathbb{R}$ neka konstanta.
	Tada vrijedi
	\begin{flalign*}
		(i) \quad & \konvpv{cX_n}{ca} \ , & \\
		(ii) \quad & \konvpv{X_n + Y_n}{a+b} \ , &\\
		(iii) \quad & \konvpv{X_n Y_n}{a b} \ , &\\
		(iv) \quad & \konvpv{\frac{X_n}{Y_n}}{\frac{a}{b}}
		\quad \text{za } b \neq 0 \ .
		& \\
	\end{flalign*}
\end{pro}

Za nizove slučajnih vektora $(\tsvek{X}_n)$ i $(\tsvek{Y}_n)$ kažemo da su 
$\be{asimptotski ekvivalentni}$
\index{asimptotska ekvivalencija}
ako vrijedi $\konvpv{(\tsvek{X}_n-\tsvek{Y}_n)}{0}$ kada $\konv{n}{\infty}$.

Čest problem u teoriji velikih uzoraka (eng. \emph{large sample theory}) je sljedeći:
za dani niz slučajnih vektora $(\tsvek{X}_n)$ i dani limes po distribuciji $\tsvek{X}$ 
($\konvpd{\tsvek{X}_n}{\tsvek{X}}$) 
treba odrediti limes po distribuciji od $(f(\tsvek{X}_n))$ za neku
funkciju $f$. Sljedeći teoremi prezentiraju neke od rezultata
kojima se rješavaju takvi problemi.

\begin{tm} \label{tm:konvpdfunkcijavektora}
	{\hspace{\stretch{1}}} \nopagebreak
	\begin{itemize}
		\item[(i)]{Ako vrijedi $\tsvek{X}_n \in \mathbb{R}^d$ i 
			$\konvpd{\tsvek{X}_n}{\tsvek{X}}$
			te je funkcija $f:\setR^d \rightarrow \mathbb{R}^k$
			takva da $\vP(\tsvek{X} \in C(f)) = 1$ 
			tada slijedi $\konvpd{f(\tsvek{X}_n)}{f(\tsvek{X})}$}
		\item[(ii)]{Ako $\konvpd{\tsvek{X}_n}{\tsvek{X}}$ i
			$\konvpv{(\tsvek{X}_n-\tsvek{Y}_n)}{0}$ 
			tada
			$\konvpd{\tsvek{Y}_n}{\tsvek{X}}$.}
		\item[(iii)]{Ako vrijedi $\tsvek{X}_n \in \mathbb{R}^d$,
			$\tsvek{Y}_n \in \mathbb{R}^k$,
			$\konvpd{\tsvek{X}_n}{\tsvek{X}}$,
			$\konvpd{\tsvek{Y}_n}{\tv{c}}$ tada
			\begin{equation}
				\konvpd{
					\begin{pmatrix}
						\tsvek{X}_n \\
						\tsvek{Y}_n \\
					\end{pmatrix}
				}{
					\begin{pmatrix}
						\tsvek{X} \\
						\tv{c} \\
					\end{pmatrix}
				}
			\end{equation}
		}
	\end{itemize}
\end{tm}

U prethodnom teoremu s
$
\begin{pmatrix}
	\tsvek{X}_n \\
	\tsvek{Y}_n \\
\end{pmatrix}
$
označili smo vektor iz $\mathbb{R}^{d+k}$ kojem je prvih $d$ elemenata
dano s $\tsvek{X}_n$, a zadnjih $k$ s $\tsvek{Y}_n$, te sa $\tv{c}$
neki koncentrirani slučajni vektor u $\setR^k$.

\begin{primjer_}
	Neka je $\tsvar{X} \sim N(0,1)$ i $\konvpd{\tsvar{X}_n}{\tsvar{X}}$.
	Tada za funkciju
	$f(x)=x^2$ po Teoremu \ref{tm:konvpdfunkcijavektora}
	(dio (i)) slijedi $\konvpd{\tsvar{X}_{n}^{2}}{\tsvar{X}^2}$ budući da je
	$f$ neprekidna. Budući je $\tsvar{X}^2 \sim \chi_{1}^{2}$ kada je
	$\tsvar{X} \sim N(0,1)$ dobivamo $\konvpd{\tsvar{X}_{n}^{2}}{\chi_{1}^{2}}$.
\end{primjer_}

\begin{primjer_}
	Pogledajmo primjer gdje prekidnost funkcije stvara probleme.
	Neka je $\tsvar{X}_n = \sfrac{1}{n}$ i 
	\[
		f(x) = 
		\begin{cases}
			1, & x>0 \\
			0, & x \leq 0
		\end{cases}
	\]
	Tada $\konvpd{\tsvar{X}_n}{0}$, ali $\nkonvpd{f(\tsvar{X}_n)}{f(0)}$.

\end{primjer_}

\begin{primjer_}
	Dio (iii) teorema \ref{tm:konvpdfunkcijavektora} ne može se
	unaprijediti pretpostavljajući da $\konvpd{\tsvek{Y}_n}{\tsvek{Y}}$ 
	te zaključivši
	$
	\konvpd{
		\begin{pmatrix}
			\tsvek{X}_n \\
			\tsvek{Y}_n \\
		\end{pmatrix}
	}{
		\begin{pmatrix}
			\tsvek{X} \\
			\tsvek{Y} \\
		\end{pmatrix}
	}
	$.
	Uzmimo $\tsvar{X} \sim U([0,1])$ i $\tsvar{X}_n=\tsvar{X}$
	za svaki $n \in \mathbb{N}$,
	$\tsvar{Y}_n = \tsvar{X}$ za $n$ neparan te $\tsvar{Y}_n=1-\tsvar{X}$
	za $n$ paran. Tada
	$\konvpd{\tsvar{X}_n}{\tsvar{X}}$ i 
	$\konvpd{\tsvar{Y}_n}{U([0,1])}$, ali
	$
		\begin{pmatrix}
			\tsvar{X}_n \\
			\tsvar{Y}_n \\
		\end{pmatrix}
	$
	ne konvergira po distribuciji.
\end{primjer_}

Izrazito važna posljedica teorema \ref{tm:konvpdfunkcijavektora}
dana je sljedećim korolarom.

\begin{kor}
	\label{kor:konvpdzajednickogvektora}
	Ako vrijedi $\tsvek{X}_n \in \mathbb{R}^d$,
	$\konvpd{\tsvek{X}_n}{\tsvek{X}}$,
	$\tsvek{Y}_n \in \mathbb{R}^k$, 
	$\konvpd{\tsvek{Y}_n}{\tv{c}}$
	te je funkcija $f:\mathbb{R}^{d+k} \rightarrow \mathbb{R}^r$
	takva da 
	$\vP \left(
		\begin{pmatrix}
			\tsvek{X} \\
			\tv{c} \\
		\end{pmatrix}
		\in C(f) \right) = 1$ tada slijedi 
	$\konvpd{f(\tsvek{X}_n,\tsvek{Y}_n)}{f(\tsvek{X},\tv{c})}$.
\end{kor}

\begin{primjer_}
	Ako $\konvpd{\tsvek{X}_n}{\tsvek{X}}$ i 
	$\konvpd{\tsvek{Y}_n}{\tv{c}}$ iz korolara
	\ref{kor:konvpdzajednickogvektora}
	slijedi $\konvpd{\tsvek{Y}_{n}^{T} \tsvek{X}_n}{\tv{c}^{T} \tsvek{X}}$ budući da je
	skalarni produkt neprekidna funkcija.
\end{primjer_}

Kod konvergencije po vjerojatnosti vrijede analogna pravila kao
i kod teorema \ref{tm:konvpdfunkcijavektora} osim
što se dio (iii) može pojačati.

\begin{tm}
\label{tm:slutsky}
%\index{Teorem!Slutsky}
	{\hspace{\stretch{1}}} \nopagebreak
	\begin{itemize}
		\item[(i)]{Ako vrijedi $\tsvek{X}_n \in \mathbb{R}^d$,
 			$\tsvek{X} \in \mathbb{R}^d$,
			$\konvpv{\tsvek{X}_n}{\tsvek{X}}$
			te za funkciju $f:\mathbb{R}^d \rightarrow \mathbb{R}^k$
			vrijedi 
			$\vP(\tsvek{X} \in C(f)) = 1$ 
			tada slijedi $\konvpv{f(\tsvek{X}_n)}{f(\tsvek{X})}$.}
		\item[(ii)]{Ako $\konvpv{\tsvek{X}_n}{\tsvek{X}}$
			i $\konvpv{(\tsvek{X}_n-\tsvek{Y}_n)}{0}$ 
			tada $\konvpv{\tsvek{Y}_n}{\tsvek{X}}$.}
		\item[(iii)]{Ako vrijedi $\tsvek{X}_n \in \mathbb{R}^d$,
			$\tsvek{Y}_n \in \mathbb{R}^k$,
			$\konvpv{\tsvek{X}_n}{\tsvek{X}}$,
			$\konvpv{\tsvek{Y}_n}{\tsvek{Y}}$ tada
			\begin{equation}
				\konvpv{
					\begin{pmatrix}
						\tsvek{X}_n \\
						\tsvek{Y}_n \\
					\end{pmatrix}
				}{
					\begin{pmatrix}
						\tsvek{X} \\
						\tsvek{Y} \\
					\end{pmatrix}
				}
			\end{equation}
		}
	\end{itemize}
\end{tm}

Lako se može pokazati da vrijedi i
analogon teoremu \ref{tm:slutsky} za konvergenciju gotovo
sigurno.

Jedna od osnovnih klasa teorema teorije vjerojatnosti
jesu jaki i slabi zakoni velikih brojeva. Ovdje
izdvajamo dva najpoznatija.

\begin{tm}
\label{tm:zvb}
	Neka su $\tsvek{X}$, $\tsvek{X}_1$, $\tsvek{X}_2$, \ldots nezavisni jednako
	distribuirani slučajni vektori te neka je
	$\overline{\tsvek{X}_n} = \frac{1}{n} \sum_{i=1}^{n} \tsvek{X}_i$.
	Tada vrijedi
	\begin{align}
		\text{\be{jaki zakon velikih brojeva:}}
		\quad &
		\konvgs{\overline{\tsvek{X}_n}}{\vE{\tsvek{X}}}
		\Leftrightarrow
		\vE{|\tsvek{X}|} < \infty
		\\
		\text{\be{slabi zakon velikih brojeva:}}
		\quad & 
		\vE{|\tsvek{X}|} < \infty
		\implies
		\konvpv{\overline{\tsvek{X}_n}}{\vE{\tsvek{X}}}
	\end{align}
\end{tm}

\noindent
Vrijednost 
$\overline{\tsvek{X}_n} = \frac{1}{n} \sum_{i=1}^{n} \tsvek{X}_i$
često nazivamo
\be{uzoračkim očekivanjem}.
\index{uzoračko očekivanje}

Opširniji pregled poznatih rezultata,
kao i dokazi odgovarajućih tvrdnji, mogu
se naći u ~\textcite{ferguson_course_1996},
\textcite{sarapa2002vjerojatnost},
\textcite{shiryaev1995probability}
ili
\textcite{allan2012probability}.
