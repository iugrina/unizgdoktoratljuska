\section{Uvod}

Riječ palindrom uvriježena je među različitim
nacijama kao izraz za niz znakova koji se
čitaju isto slijeva nadesno i zdesna nalijevo.
Prema hrvatskom jezičnom portalu (\cite{hjp}) definicija bi bila
{\glqq}igra riječi u kojoj se čitanjem jedne riječi ili čitave rečenice 
obrnutim redom dobiva isto značenje kao i pravilnim čitanjem''.
Etimologija same riječi dolazi
iz grčkog \emph{palindromos}: koji trči natrag
(\emph{palin-} + \emph{-drom}).
Vjerojatno je većini poznat iz mozgalica kojima
se djeca zabavljaju u osnovnoj školi poput
{\glqq}Ana voli Milovana''.

Osnovno proširenje pojma palindroma
dolazi iz područja bioinformatike gdje se palindromi
definiraju kao nizovi znakova koji se čitaju slijeva
isto kao i tome komplementarni niz zdesna. 
Komplementarni niz
se dobiva zamjenom slova klasičnim povezivanjem
adenina, citozina, guamina i timina. Odnosno,
\nzdna{A} se mijenja sa \nzdna{T} (i obratno) te
\nzdna{C} sa \nzdna{G} (i obratno). Primjer palindroma u DNA
nizu jest \nzdna{ACTAGT} jer je njegov komplementarni niz
\nzdna{TGATCA}. Lako je uočiti 
da su jedini mogući palindromi
u DNA nizovima oni parne duljine.

Korisnost proučavanja palindroma u DNA nizovima slijedi iz dosadašnjih
istraživanja koja ukazuju na zaključak da su palindromi
u molekuli DNA nužni za funkcioniranje genoma jer se često nalaze u 
bis-djelujućim regijama, ali istovremeno predstavljaju ozbiljnu opasnost
za genetičku stabilnost
(\textcite{lobachev_hairpin-_2007}),
pri čemu genetička nestabilnost palindroma raste 
s njihovom duljinom
(\textcite{nag_140-bp-long_1997}).
Palindromi se nalaze i kao dijelovi regulacijskih sekvenci,
poput operatora
(\textcite{sinden_perfect_1983}),
promotora
(\textcite{thukral_two_1991}),
terminatora 
(\textcite{gomes_evolution_2007})
i ishodišta replikacije
(\textcite{chew_scoring_2005})
te stoga čine nužan dio genetičkog materijala.
Također, zbog definirane simetrije palindromi u DNA nizovima
tvore sekundarne strukture poput strukture ukosnice
čije formiranje tijekom 
replikacije može uzrokovati {\glqq}proklizavanje''
DNA-polimeraze te može zaustaviti replikaciju
(\textcite{waldman_long_1999}).

Višeznačnost palindroma u DNA nizovima potaknula je razna
bioinformatička istraživanja sekvencioniranih genoma s ciljem 
ustanovljivanja broja, duljine i rasporeda svih palindroma. 

Velik dio istraživanja sastoji se od prepoznavanja
sadrži li niz značajno 
više palindroma od očekivanog broja
(\textcite{robin_dna_2005})
pa bi poznavanje distribucije
broja palindroma u unaprijed definiranom vjerojatnosnom
modelu DNA niza bilo od velike koristi.

Osim u bioinformatici palindromi su standardni dio
istraživanja i u diskretnoj matematici
(područje koje se zove kombinatorika nad riječima)
gdje, na primjer, igraju važnu ulogu pri generiranju
Christoffelovih riječi
s primjenama u teoriji brojeva 
(diofantskim aproksimacijama)
(\textcite{lothaire_applied_2005,
fischler_palindromic_2006}).
