\section{Povezana istraživanja}
\label{pal:sec:referencework}

Vjerojatnosna svojstva različitih oblika nizova znakova
(riječi) učestali su predmet istraživanja posljednja
dva desetljeća. Povećanjem računalne moći te naročito
probojem osobnih računala analiza nizova znakova,
u prvom redu DNA nizova, postaje moguća te time
i zanimljiva.

Značaj palindroma za različite biološke nizove,
kao što je prikazano u Uvodu, dovodi do potrebe
za kvalitetnim alatima pri analiziranju svojstava
palindroma kao i odnosa između palindroma i 
nizova znakova u kojima se nalaze. Budući
da su palindromi ustvari vrste riječi s posebnom
strukturom, rezultate o vjerojatnosnim svojstvima
riječi većinom je moguće izravno primijeniti
i na palindrome.

U poglavlju {\glqq}Statistike nad riječima s primjenama
na biološkim nizovima'' knjige 
\cite{lothaire_statistics_2005}[poglavlje 6]
izvode se određena svojstva riječi pa time i palindroma.
Za niz znakova pretpostavlja se da je generiran 
homogenim Markovljevim lancem proizvoljnog reda
(n.j.d. slučaj je reda 0)
unoseći time zavisnost među
elementima. Međutim, ista se zavisnost podrazumijeva
za cijeli niz pa se time gubi ideja regija, odnosno
nije moguće modelirati dijelove niza znakova kao
različite regije s različitim distribucijama.
Među prezentiranim rezultatima bitno je spomenuti
da su izneseni rezultati o asimptotskoj distribuciji
broja palindroma kao i odgovarajućim procjeniteljima
za očekivanje i varijancu. Također, dane su ocjene 
kvalitete aproksimacija normalnom i Poissonovom
distribucijom koristeće Steinovu i Chen-Steinovu
metodu. Svi su rezultati dani za homogene modele,
a autori navode da do trenutka pisanja knjige
nisu upoznati sa statističkim rezultatima
nad heterogenim modelima.

Knjiga \textcite{robin_dna_2005} prezentacija je dijela
rezultata (tehnički manje zahtjevnih) spomenutih i u 
knjizi \cite{lothaire_statistics_2005}
uz dodatak ilustracije važnosti aproksimacija broja
palindroma kroz uvod u bioinformatiku.

\textcite{chew_scoring_2005} promatraju problem
određivanja klastera unutar DNA nizova s velikim
(neočekivanim) brojem palindroma.
Kao pomoć pri klasteriranju definiraju tri
različite težine palindroma dajući tim
određenim palindromima prednost nad drugima.
U radu se pretpostavlja homogeni model DNA niza.


\textcite{leung_nonrandom_2005} također promatraju problem određivanja
iznimnih klastera palindroma. U radu se daje dokaz nekih
osnovnih svojstava palindroma u DNA nizovima te
se izvodi aproksimacija pojavljivanja palindroma
Poissonovim procesom za n.j.d. model DNA niza
uz uvjet $p_A=p_T$ i $p_C = p_G$.

Distribucija broja palindroma po različitim regijama
ljudskog genoma empirijski je određena
u \textcite{lu_human_2007}.

U \textcite{chan_importance_2010} prikazana je primjena
računalnih simulacija, definirani su specifični algoritmi,
na određivanje distribucije broja riječi (ne nužno palindroma).

\textcite{robin_numerical_2001} daju numeričku usporedbu
nekoliko aproksimacija distribucije broja riječi
(međusobno i naspram egzaktne distribucije)
u slučajnim nizovima znakova.
Korištene su aproksimacije normalnom i 
složenom Poissonovom razdiobom.
Pretpostavlja se homogeni Markovljev model kao
model niza znakova.

\textcite{zhai_normal_2012} izvode aproksimaciju 
normalnom i složenom Poissonovom razdiobom 
za distribuciju broja riječi kod
NGS (Next generation sequencing) tehnika modeliranja
nizova znakova.

U diplomskom radu \textcite{gacesa_zastupljenost_2011} razvijen je računalni program
za određivanje očekivanog broja palindroma u kodirajućoj DNA. Program se zasniva
na uspoređivanju broja palindroma u zadanom nizu i u računalno generiranim 
nizovima koji se od zadanog niza razlikuju samo po položaju sinonimnih kodona.

Autor ovog doktorskog rada, u koautorstvu s D. Špoljarićem, 
objavio je sažetu verziju
rezultata prezentiranih u ovom poglavlju kao dio rada
\textcite{spoljaric_statistical_2013}.

