\subsection{Definicija modela}
\label{pal:sec:defmodela}

Neka su slučajne varijable $\tsvar{X}_n$
nezavisne i distribuirane na sljedeći način:
\[
	\tsvar{X}_n \sim
	\begin{pmatrix}
	a_1 & a_2 & a_3 & \cdots & a_K \\
	p_{1}^{(n)} & p_{2}^{(n)} & p_{3}^{(n)} & \cdots & p_{K}^{(n)} \\
	\end{pmatrix}
	\ ,
\]
odnosno $\vP (\tsvar{X}_n = a_k) = p_{k}^{(n)}$ za sve $n \in \setN$,
$k \leq K$. Vrijednosti
$a_i$ predstavljaju znakove neke abecede, a $p_{i}^{(n)}$ vjerojatnosti
pojavljivanja tih znakova na mjestu $n$ u nizu. Skup svih znakova $a_i$
označavat ćemo sa $\mathcal{A} = \{a_1, a_2, \ldots, a_K \}$.

Neka je $\ol{\cdot} : \mathcal{A} \rightarrow \mathcal{A}$ 
bijektivna funkcija
takva da vrijedi $\dol{x} = x $ za svaki $x \in \mathcal{A}$.
Takvu ćemo funkciju nazivati
\be{funkcijom komplementarnosti}.
\index{funkcija!komplementarnosti}
Ukoliko budemo koristili indekse kao u izrazu $\ol{\tsvar{X}_{i+j}}$
često ćemo pisati samo $\ol{\tsvar{X}}_{i+j}$.
Najjednostavniji primjer funkcije komplementarnosti jest identiteta.

\begin{defn}
	\label{df:palindrom}
	Za niz znakova $b_1$, $b_2$,\ldots, $b_M$ iz $\mathcal{A}$
	kažemo da tvori
	\begin{itemize}
		\item[(i)]{
			\be{(paran) palindrom}
			\index{palindrom}
			\index{palindrom!paran}
			veličine $M$ ako je $M$ paran broj te vrijedi
			$$
			b_1 = \ol{b}_M,\ b_2 = \ol{b}_{M-1}, \ldots,
			b_{\frac{M}{2}} = \ol{b}_{\frac{M}{2} +1}
			\ ,$$}
		\item[(ii)]{
			\be{neparan palindrom}
			\index{palindrom!neparan}
			veličine $M$ ako je $M$ neparan broj te vrijedi
			$$
			b_1 = \ol{b}_M,\ b_2 = \ol{b}_{M-1}, \ldots,
			b_{\frac{M-1}{2}} = \ol{b}_{\frac{M+3}{2}},
			b_{\frac{M+1}{2}} = \ol{b}_{\frac{M+1}{2}}
			\ ,$$}
	\end{itemize}
\end{defn}

U ostatku teksta proučavat ćemo parne palindrome te u
rijetkim dijelovima i neparne palindrome.

\begin{napomena_}
	\label{nap:neparnipalindromidentiteta}
	Iz definicije neparnog palindroma vidimo da za centralni
	element mora vrijediti $b_{\frac{M+1}{2}} = \ol{b}_{\frac{M+1}{2}}$.
	Stoga, broj mogućih kombinacija za neparne palindrome
	ovisi o bliskosti funkcije komplementarnosti 
	identiteti $\ol{a}_i = a_i$ za svaki $a_i \in \mathcal{A}$.
	Odnosno, ovisi o broju elemenata od $a_i \in \mathcal{A}$ takvih
	da vrijedi $\ol{a}_i = a_i$.
\end{napomena_}

\begin{primjer_}[{DNA}]
	Poznato je da se DNA nizovi sastoje od četiri slova (A-adenin,
	C-citozin, G-guanin, T-timin) te da se prirodno uparuju
	adenin s timinom te citozin s guaninom.
	U uvodu smo opisali da će palindromi od značaja u DNA nizovima
	biti dani funkcijom komplementarnosti
	$\ol{A}=T$, $\ol{C}=G$, $\ol{G}=C$, $\ol{T}=A$,
	odnosno prirodnim uparivanjem baza.
	Iz napomene \ref{nap:neparnipalindromidentiteta} slijedi
	da se u DNA nizovima ne mogu dobiti neparni palindromi.
	Primjeri parnih palindroma u DNA nizu dani su
	u tablici \ref{tab:primjeripalindromaudna}.
	\begin{table}[!h] 
		\caption{Primjeri palindroma u DNA nizovima}
		\label{tab:primjeripalindromaudna}
		\centering
		\begin{tabular}{rl} \hline
			\nzdna{ACGT} & parni palindrom duljine 4  \\
			\nzdna{ACCGTACGGT} & parni palindrom duljine 10  \\
			\hline
		\end{tabular}
	\end{table}
\end{primjer_}

Definirajmo
\be{blok}
\index{blok!znakova}
znakova kao neprekinuti niz znakova duljine barem jedan.
Točnije, blokom $B$ duljine $L_B$ smatrat ćemo bilo koji niz
slučajnih varijabli oblika $(X_k,\cdots,X_{k+L_B-1})$
te ćemo zbog jednostavnosti koristiti oznaku $X_k\cdots X_{k+L_B-1}$.
Promatrat će se blokovi koji su disjunktni i povezani
(ne postoje elementi između blokova)
te će slučajne varijable unutar svakog bloka biti
jednako distribuirane.

Dva oblika povezivanja blokova su dopuštena
s obzirom na povećanje niza $X_1,\ldots,X_n$ (po $n$):
\begin{itemize}
\item[(T1)]{
	blokovi se mijenjaju zajedno s duljinom
	niza $n$ zadržavajući omjere veličina blokova
	unutar cijelog niza jednakima 
	\begin{table}[!htpb]
	\centering
	\begin{tabular}{|ccc|ccccccc|c|ccccc|}
	  &  $B_1$ &  &  & &  &  $B_2$ &  &  &  &  $B_3$ &  &  &  $B_4$ &  &  \\
	\hline
	\end{tabular}
\end{table}
}
\item[(T2)]{
	blokovi se izmjenjuju periodično, npr. $B_{1}B_{2}B_{3}B_{1}B_{2}B_{3}\ldots$
}
\end{itemize}
Ideja blokova proizlazi iz potrebe da se opišu različiti dijelovi
DNA nizova (npr. \glslink{dnacodingregion}{kodirajuće regije}
i \glslink{dnanoncodingregion}{nekodirajuće regije}) koji
mogu utjecati na broj palindroma budući da distribucije baza
mogu biti različite u različitim blokovima.
Na primjer, regija s visokom frekvencijom baza $A$ i $T$
ima veću vjerojatnost sadržavati velik broj palindroma
od regije s uniformnom razdiobom baza.

U ostatku doktorskog rada podrazumijevat ćemo da su veličine blokova
veće od duljine palindroma od interesa. U primjenama
se takva pretpostavka podrazumijeva budući da su palindromi
od interesa uvijek manji od regija koje opisuju blokovi
(\textcite{lisnic_palindrome_2005}).

Označimo sa $m$ fiksnu duljinu palindroma. Neka slučajna varijabla
$\tsvar{Y}_i^{(m)}$ bude indikator da palindrom duljine $m$ završava na poziciji
(indeksu) $i$ u nizu $\tsvar{X}_1\cdots\tsvar{X}_n$. Odnosno, za $\tsvar{Y}_i$ vrijedi
\begin{equation}
	\label{df:palindromslvar}
	\tsvar{Y}_i^{(m)}=
	\mathbbm{1}_{\{ \ol{\tsvar{X}}_{i} = \tsvar{X}_{i-m+1}\}} \cdot
	\mathbbm{1}_{\{ \ol{\tsvar{X}}_{i-1} = \tsvar{X}_{i-m+2}\}} \cdots
	\mathbbm{1}_{\{ \ol{\tsvar{X}}_{i-\frac{m}{2}+1} = \tsvar{X}_{i-\frac{m}{2}}\}} 
	\ .
\end{equation}

Budući da je duljina palindroma jednaka $m$, slučajne varijable
$\tsvar{Y}_i^{(m)}$ dobro su definirane za $i \geq m$.
Definirajmo sada \be{broj palindroma} duljine $m$ kao slučajnu varijablu
\begin{equation}
	\label{df:brojpalindroma}
	\tsvar{N}_n = \sum_{i=m}^{n} \tsvar{Y}_i^{(m)} \ .
\end{equation}

Ukoliko ubuduće iz konteksta bude jasno o kojoj se duljini palindroma
radi, indeks $m$ će se izostavljati iz $\tsvar{Y}_i^{(m)}$ te će se pisati
samo $\tsvar{Y}_i$.
