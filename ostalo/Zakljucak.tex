\chapter{Zaključak}

U potpoglavlju \ref{pal:sec:primjenanastvarnim} pokazano je kako
izbor modela DNA niza (n.j.d., blokovi različitih veličina)
može utjecati na zaključak, odnosno rezultate statističkog testiranja.
Stoga je važno koristiti ekspertno znanje
pri definiciji blokova kao regija te tumačiti rezultate
s obzirom na taj model.

Budući da nije jednostavno odrediti točne granice kada bi
se trebala koristiti aproksimacija normalnom razdiobom,
a kada Poissonovom ili složenom Poissonovom,
preporučljivo je voditi se 
pravilom da visok očekivani broj palindroma
više odgovara aproksimaciji normalnom razdiobom,
a nizak aproksimaciji Poissonovom razdiobom.
Ovakava se preporuka opravdava rezultatima simulacija
opisanih u ovom doktorskom radu kao i prijašnjim istraživanjima.
U posebnim slučajevima, npr. jako dug niz, mogu se koristiti
i granice ocjena pogrešaka aproksimacije danim distribucijama.

