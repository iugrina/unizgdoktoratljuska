\chapter*{Summary}
\markboth{Summary}{}

\begin{center}
	\textbf{Keywords}: 
	central limit theorem;
	m-dependent sequence;
	normal distribution;
	palindromes in DNA;
	string similarity;
	postal addresses;
	address	extraction;
	decision trees;
	geographic location;
	CP decomposition;
	outliers
\end{center}

\vspace{3ex}

In the first part of this thesis a result about the distribution of the number 
of palindromes of a fixed length in a sequence of characters (string) is presented. 
Palindrome is defined as a part of the string which is equal to its complementary
sequence read backwards. Complementarity is defined by some relation of characters
(e.g. in DNA sequences the natural relation is 
\nz{C}$\sim$\nz{G}, \nz{A}$\sim$\nz{T}). 
The general case where the distribution of the characters is arbitrary is presented. 
Further, conditions in which Normal approximation can be used are derived.
Special attention is given to modeling of coding and non-coding
regions in DNA and to the distribution of bases in those parts of DNA.
An example of an application of results to a real life sequence is also presented.

